\chapter{Összefoglalás}
\label{chap:osszefoglalas}
Dolgozatomban a feladatnak megfelelően bemutattam az állapottérképek formalizmusát és megterveztem egy folyamatot melynek segítségével támogatni lehet ezek formális verifikációját a MagicDraw modellező eszközben. A tervezett eszköz prototípusát megvalósítottam egy plug-in formájában melynek a működését egy esettanulmányon demonstráltam, végül pedig értékeltem az elvégzett munkát és megvizsgáltam a továbbfejlesztési lehetőségeket.

\paragraph{Az elvégzett munka pontokba szedve:}

\begin{itemize}
	\item Megfeleltetések megtervezése
	\begin{itemize}
		\item Összevetettem a két eszköz Meta-modelljét
		\item Kiválasztottam az egymásnak megfeleltethető elemeket
		\item Odafigyeltem a szemantikai különbségekre
	\end{itemize}
	\item Leképzés implementációja
		\begin{itemize}
		\item modell transzformációkra specializált technológiát használtam
		\item az eszközt MagicDraw plug-in formájában valósítottam meg
	\end{itemize}
	\item Fejlesztőkörnyezet kialakítása
		\begin{itemize}
			\item összegyűjtöttem a szükséges dependenciákat
			\item odafigyeltem a tranzitív dependenciák helyes menedzselésére
		\end{itemize}
	\item Lehetővé tettem, hogy a MagicDraw-n belül elvégezhető legyen a verifikáció
			\begin{itemize}
			\item átvettem és módosítottam a Gamma Query Generátor funkcióját
		\end{itemize}

	\item Elméleti megközelítésből értékeltem a munkám
		\begin{itemize}
			\item esettanulmányon mutatom be az elkészült eszközt
			\item áttekintetem az alternatív megvalósítási lehetőségeket
		\end{itemize}
	%\item mérés
	
\end{itemize}

Munkám eredményéül létrejött egy olyan MagicDraw plug-in amivel lehetőség nyílik állapottérképek formális verifikációjának végrehajtásába.

\section{Jövőben elvégzendő munka}

Az eddigi munkám során létrejött eszköz továbbfejleszthető, hogy lehetővé tegye komponens alapú modellek verifikációját is. A jövőben a leképzéseket kiterjesztem, hogy ezeket a modelleket is Gamma modellé lehessen transzformálni és elvégezni a formális verifikációt a rendszeren. Továbbá megtervezek egy olyan új funkciót ami megjeleníti a felhasználóknak azokat az utakat, amik a megszorításaik megsértéséhez vezettek. A felhasználói élmény javítása érdekében validációs szabályokat hozok létre, amik futásidőben figyelmeztetik a felhasználókat azoknak az elemeknek a használatára, amelyek felhasználása nem teszi lehetővé a leképzést, vagy a formális verifikáció végrehajtását. Továbbá megfontolom olyan elemek támogathatóságát, amik az állapottérképek újrahasznosítását teszik lehetővé (\emph{SubmachineState}).


%tervezett