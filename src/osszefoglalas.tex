\chapter{Összefoglalás}
\label{chap:osszefoglalas}
%fleadat lap pontjainak összefoglalása

%pontokba szedve kontribúciók

\begin{itemize}
	\item Megfeleltetések megtervezése
	\begin{itemize}
		\item Összevetettem a két eszköz Meta-modelljét
		\item Kiválasztottam az egymásnak megfeleltethető elemeket
		\item Odafigyeltem a szemantikai különbségekre
	\end{itemize}
	\item Leképzés implementációja
		\begin{itemize}
		\item modell transzformációkra specializált technológiát használtam
		\item az eszközt MagicDraw plug-in formájában valósítottam meg
	\end{itemize}
	\item Fejlesztőkörnyezet kialakítása
		\begin{itemize}
			\item összegyűjtöttem a szükséges dependenciákat
			\item odafigyeltem a tranzitív dependenciák helyes menedzselésére
		\end{itemize}
	\item Lehetővé tettem, hogy a MagicDraw-n belül elvégezhető legyen a verifikáció
			\begin{itemize}
			\item átvettem és módosítottam a Gamma Query Generátor funkcióját
		\end{itemize}

	\item Elméleti megközelítésből értékeltem a munkám
		\begin{itemize}
			\item esettanulmányon mutatom be az elkészült eszközt
			\item áttekintetem az alternatív megvalósítási lehetőségeket
		\end{itemize}
	%\item mérés
	
\end{itemize}

Munkám eredményéül létrejött egy olyan MagicDraw plug-in amivel lehetőség nyílik állapottérképek formális verifikációjának végrehajtásába.

\section{Jövőben elvégzendő munka}
\begin{enumerate}
	\item Komponens alapú modellezés támogatása
	\item Megkötések megsértéséhez vezető utak megjelenítése
	\item Leképezhető elemek validációja
	\item Újrahasznosíthatóság támogatása (\emph{Submachine State})
\end{enumerate}


%tervezett