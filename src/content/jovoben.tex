\chapter{Továbbfejlesztési lehetőségek}

\section{IBD - Composition Language}
A Gamma Statechart Composition Framework egyik legfontosabb funkciója, hogy lehetővé teszi állapottérképekből, mint komponensekből egy komplett rendszer leírását. Ilyesfajta leírás SysML-ben az Internal Block Diagram(IBD).

Az IBD-k leképzésének támogatásával a felhasználók képessé válnának komplex reaktív rendszereket leírni és verifikálni.

\section{Szimuláció generálása}

Az UPPAAL opcionálisan előállítja azokat az utakat melyek sértik a megkötéseket. A Gamma Framework képes ezekből kódot és Yakindu szimulációt előállítani.

A MagicDraw is rendelkezik egy szimulátorral Cameo Simulation Toolkit\footnote{Cameo simulation toolkit: https://www.nomagic.com/product-addons/magicdraw-addons/cameo-simulation-toolkit} néven. A Cameo szimulátor plug-in a No Magic terméke. Segítségével modelleket lehet debugolni, szimulálni és UI prototyping funkcionalítással rendelkezik.

A szimulációkat modell elemekkel is fel lehet konfigurálni Execution Configuration Classok segítségével, ezért potenciálisan modell transzformációkkal elő lehet állítani szimulációt.

\section{Validation kit}

A VIATRA-va inkrementális és reaktív tulajdonsági lehetővé teszik modell transzformációk végrehajtását, ha a modellt változik. Ezt a funkcionalitást kihasználva lehetőséget kapunk, hogy létrehozzuk validációs szabályok VQL-ben leírt halmazát és ezeket futás időben folyamatosan ellenőrizve a MagicDraw API-ján keresztül felannotálhatjuk azokat az elemeket amik nem felelnek meg a ezeknek szabályoknak. Ezeket a MagicDraw megjeleníti a GUI-ján.

Ezek a validációs szabályok lehetnek figyelmeztetések, hogy melyik elemek nem képezhetőek le, vagy leképezhetőek de nem támogatott a verifikációjuk, ezzel növelve a felhasználói élményt, hogy ne a transzformációk végrehajtása alatt értesüljenek a potenciális hibákról.




