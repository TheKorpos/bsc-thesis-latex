\chapter{Összefoglalás}

Dolgozatomban a feladat kiírásnak megfelelően bemutattam az eszköz korábbi változatát, majd ismertettem azokat a módosításokat melyeket a szoftveren végeztem. A módosítások eredményeképp már kompozit, hierarchikus állapottérképek formális verifikációjára is lehetőség van, az ellenpéldák a modellben létrejönnek. A továbbfejlesztett eszköz működését bemutattam egy példán és mérésekkel vizsgáltam a beépülő modul teljesítményét.

\paragraph{A fontosabb kontribúciók:}
\begin{itemize}
	\item Komponensek transzformációja
		\begin{itemize}
			\item UML profil a komponens szemantika módosításához
			\item Támogatott modell struktúrák
			\item A transzformáció megvalósítása
			\item Validációs készlet tervezése
		\end{itemize}
	
	\item Példák, ellenpéldák megjelenítése
		\begin{itemize}
			\item UML profil a szemantika változtatásához
			\item Ellenpélda back-annotálása a SysML modellbe
			\item Szcenárió megjelenítésének vizsgálata különböző diagramtípusokon
			\item Kísérletek a szcenáriók végrehajtására a szimulátorral
		\end{itemize}
	\item Property nyelvtan felhasználása
		\begin{itemize}
			\item tulajdonságok definiálása a Gamma Property nyelvtanán speciális modell elemekben	
		\end{itemize}
	\item Megvalósítás értékelése teljesítmény szempontjából
		\begin{itemize}
			\item transzformáció végrehajtási idejének mérése
			\item a VIATRA minták memória igényének vizsgálata	
		\end{itemize}
\end{itemize}
A munkám eredményeként komponens és állapot alapú modelleken is végre lehet hajtani a formális verifikációt. A modellben back-annotáció formájában megjelennek a példák és ellenpéldák. A rendszer tulajdonságait pedig a Gamma \emph{Property} nyelvtanának segítségével lehet megfogalmazni.

\section{Lehetőségek az eszköz továbbfejlesztésére}

Az ellenpéldák megjelennek a modellben, de a szcenáriókat leíró diagramok automatikusan nem állnak elő. Ezek előállítása szintén segítené a mérnököket a modellek vizsgálatában. Ezen felül egy szimulátor motort is létre lehetne  hozni, ami lehetővé tenné az események helyes visszajátszását.

Ahogy arra a mérések is rámutattak van olyan megoldás aminek az alkalmazása teljesítmény problémához vezetett. Ezek kijavítása vagy más megoldásra cserélése a jövőben indokolt.


