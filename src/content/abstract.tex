\pagenumbering{roman}
\setcounter{page}{1}

\selecthungarian

%----------------------------------------------------------------------------
% Abstract in Hungarian
%----------------------------------------------------------------------------
\chapter*{Kivonat}\addcontentsline{toc}{chapter}{Kivonat}

%kontextus
Komplex rendszerek tervezése során szükségessé válik hogy fejlett modellező eszközök álljanak rendelkezésre, hiszen gyakran szükség van ezen alkalmazások különböző funkcióinak igénybevételére, pl. validációra, kódgenerálásra, verifikációra. Az iparban az egyik legelterjedtebb eszköz a MagicDraw ami lehetőséget biztosít rendszerek tervezésére és ezek viselkedésének modellezésére. MagicDraw-t gyakran használnak kritikus rendszerek tervezésére, melyek esetén elengedhetetlen, hogy megbizonyosodjunk a rendszer működésének helyességéről.

%probléma
Bár a MagicDraw már sok hasznos funkcióval rendelkezik, azonban a formális verifikációt nem támogatja. Az ipari alkalmazhatóság szempontjából jelentős előnyökkel járna ezeknek a módszereknek a támogatása.
%megoldás
A Gamma egy a tanszéken fejlesztett modellező eszköz, amely nem rendelkezik a MagicDraw-éhoz hasonlóan kiterjedt eszközkészlettel, azonban viselkedésmodellek formális verifikációját támogatja. Dolgozatomban MagicDraw modellek formális verifikációját azáltal teszem lehetővé, hogy megvalósítok egy leképzést a két eszköz modelljei között.

Dolgozatomban bemutatom a Gamma és MagicDraw eszközöket és a leképzés megvalósításához szükséges háttérismereteket. Továbbá részletesen kitérek az implementáció során használt technológiákra és elvekre. A bemutatott módszert egy példán szemléltem és elméleti megfontolások alapján értékelem. Az elkészített eszköz lehetővé teszi modellek egy tágabb halmazának formális verifikációját.


\vfill
\selectenglish


%----------------------------------------------------------------------------
% Abstract in English
%----------------------------------------------------------------------------
\chapter*{Abstract}\addcontentsline{toc}{chapter}{Abstract}

The development of complex systems makes it neccessary to gain access to such modern modelling tools that support functionalities like validation, code generation and verification. One of the most well-known tool in the indsutry is MagicDraw. A tool for desinging systems and to model their behavior. MagicDraw is often used during the development of fault tolerant systems. It is mandatory to ensure that such systems operate correctly.

Although MagicDraw has very rich a functionallity it lacks the ability to perform formal verification. The use of such methods would be very beneficial for industrial use. Gamma is a modelling tool developed at the Department which is not as extensive in functions as MagicDraw is but it does support the verification of behavioral models. In my thesis on supporting formal verification of MagicDraw models I rely on performing a transformation between the models of the two tools.

The thesis introduces both Gamma and Magicdraw tools and the background knowledge needed to understand the transformation method. Furhtermore I describe in details the concepts and techologies used in my implementation. I introduce the described method via an example and I evaluete the results based on theoretical considerations. The finished product enables the formal verifications on a wider set of models.


\vfill
\selectthesislanguage

\newcounter{romanPage}
\setcounter{romanPage}{\value{page}}
\stepcounter{romanPage}