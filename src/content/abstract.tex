\pagenumbering{roman}
\setcounter{page}{1}

\selecthungarian

%----------------------------------------------------------------------------
% Abstract in Hungarian
%----------------------------------------------------------------------------
\chapter*{Kivonat}\addcontentsline{toc}{chapter}{Kivonat}


Biztonságkritikus rendszerek tervezéséhez különösen fontos, hogy rendelkezésünkre álljanak olyan eszközök, melyek segítségével meg lehet vizsgálni, hogy a rendszer modellezett viselkedése valóban megfelel a rendszerrel szemben támasztott követelményeknek. A viselkedés modellezése magas szinten gyakran történik állapot alapú formalizmusokkal, mint például a UML állapottérképekkel.

A MagicDraw egy UML és SysML modellező eszköz amely elterjedt az iparban. Ehhez az eszközhöz készült egy beépülő modul, ami lehetővé teszi logikai formulák segítségével specifikált tulajdonságok teljesülésének ellenőrzését állapottérképeken. A mérnökök gyakran használnak hierarchikus, komponens alapú állapot modelleket rendszerek modellezésére, azonban a beépülő modul ezek együttes ellenőrzésére nem képes. A tulajdonságok teljesülésének vizsgálata olyan példák keresését jelenti, amik bizonyítják ennek teljesülését. Fontos, hogy a mérnökök ezekhez hozzáférjenek és elemezni tudják. 

Dolgozatomban bemutatom a beépülő modult és hogy milyen módon fejlesztettem azt tovább, hogy lehetőséget biztosítson komponenst alapú hierarchikus állapot modellek formális ellenőrzésére és az ellenőrzés során keletkező bizonyítékként szolgáló példák megjelenítésére. Az elkészítette eszköz működését egy példán keresztül bemutatom, majd mérések segítségével értékelem a gyakorlati megvalósítás hatékonyságát.


\vfill
\selectenglish


%----------------------------------------------------------------------------
% Abstract in English
%----------------------------------------------------------------------------
\chapter*{Abstract}\addcontentsline{toc}{chapter}{Abstract}

TODO:

\vfill
\cleardoublepage

\selectthesislanguage

\newcounter{romanPage}
\setcounter{romanPage}{\value{page}}
\stepcounter{romanPage}