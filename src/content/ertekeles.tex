\chapter{Értékelés}

\section{Alternatíva - kódgenerálás}
A Gamma saját nyelvtanokkal rendelkezik, melyek Xtext segítségével vannak implementálva. Ez a technológia lehetővé teszi, hogy saját szintaxis alapján, lehessen kódot írni és EMF példánymodellt generálni a leírásból.

Ezt a mechanikát ki lehet használni a modell transzformáció alternatívájaként: a MagicDraw modellből Gamma Statechart Language szintaxisának megfelelő kódot lehetne generálni és ezt Xtext segítségével leparseolni.

A visszakövethetőség is megoldható, ehhez a nyelvtant annotációkkal kéne kibővíteni, amik jelölnék az eredeti elemeket.

Ennek előnye, hogy a kimeneteket utána tovább lehetne importálni Eclipse-be és abban folytatni a fejlesztést.

Hátránya viszont, hogy a skálázhatóságot sokkal nehezebb megoldani, és kevésbé flexibilis mint a transzformáció.

\section{Kifejezések kifejező ereje}
A MagicDrawToGamma által támogatott Guard és Action definíciók korlátozottabban használhatók a jelenlegi implementációban, mint amit egyébként a Gamma támogatni tudna, ezért célszerű lenne ezek bővítése, hogy ugyan azzal a kifejező erővel bírjanak mint a Gammában megfogalmazható párjaik.

A jelenlegi implementáció egy saját készítésű String parser ami Gamma Expressionöket állít elő. Ennek a bővítése igen nehézkes, viszont  viszonylagos egyszerűsége miatt nem jár túl nagy overhaedel és nem ad hozzá még több dependenciát a projekthez.

Alternatív megvalósításként potenciálisan lehetne használni Xtextet és a már létező nyelvtant, az ez irányba tett kísérletek, azonban a megoldás túlzott komplexitását látszanak igazolni. A komplikációk fő oka az volt, hogy a nyelvtant alkotó egy szabály szerint parse-olt kódrészletek esetében a referenciák nem oldódtak fel.

