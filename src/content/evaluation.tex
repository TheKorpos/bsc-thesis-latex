\chapter{Teljesítmény mérés, értékelés}

Ebben a fejezetben megvizsgálom az eszköz teljesítményét és azt, hogy hogyan skálázódik a funkcionalitás a modell elemszámának növekedésével. Ez azért fontos, mert  hiába van megoldásunk egy problémára, ha az a gyakorlatban az erőforrások korlátossága miatt nem képes időben eredményt adni. A két legfontosabb tényező az idő és a memóriafogyasztás. Megfelelő gyorsítótárak alkalmazásával az idő mértéke drasztikusan csökkenthető, viszont ez memóriába kerül. Ha a memória elfogy a szemét gyűjtő algoritmus agresszívabban próbálhatja összeszedni az elengedett objektumokat, ez szintén megnövelheti a végrehajtási időt, hiszen a rendszer szemétgyűjtéssel és nem pedig a működés elvégzésére használja az erőforrásokat.

A plugin VIATRA-t használ mégpedig inkrementális konfigurációval. Ez azt jelenti, hogy minták inicializációjuktól folyamatosan figyelik a modell változásait és módosítják a lekérdezések eredményhalmazát. Ez szintén memóriát vesz igényben viszont az elemek lekérdezése egy táblázat sorának kiolvasásának megfelelő komplexitással bír és nagyon gyors. Azt, hogy mennyire nő nagyra a VIATRA memória igénye leginkább a minták befolyásolják ezért különösen fontos, hogy ezek minél hatékonyabbak legyenek, mert nagy modell esetén nagyon nagy memória fogyasztást eredményezhetnek.

\section{Módszertan}

A teljesítmény mérés során az egyes transzformációk futási idejét és a VIATRA memória fogyasztását mértem. Magát a formális verifikációt  nem vizsgáltam, mert az főként az \uppaal teljesítményétől függ.



