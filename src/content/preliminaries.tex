\chapter{Előismeretek}

\section{Kapocsolódó munkák}
%TODO: 
\section{SysML}
\section{Felhasznált technológiák}
\subsection{Eclipse Modeling Framework}
\subsection{VIATRA}


\chapter{MagicDraw állapottérkép verifikációs plugin}
Ebben a fejezetben ismertetem a MagicDraw állapottérkép verifikációs eszközt, melyet korábbi egyetemen végzett tevékenységeim eredményeként jött létre és ismertetem azokat a továbbfejlesztési irányokat melyeket a mesterképzés alatt megvalósítottam.

\section{Modelltranszformációk}

Az ipari standardok mellett sok speciális modellezési nyelv is létezik, amelyek megannyi céllal és eszközkészlettel jöttek létre. Ezek között vannak magas absztrakciójú általánosabb nyelvek és alacsony szintűek is amik egy része olyan formalizmusokra épül amik felett bizonyos problémákra matematikai eszközökkel tudunk megoldást keresni.

A modellvezéreltség egyik alapötlete, hogy különböző modellekből származtatni tudunk más modelleket feltéve, hogy elegendő információ áll rendelkezésünkre a konverzió elvégzéséhez. Ezt a fajta származtatási folyamatot modelltranszformációnak nevezzük. A modelltranszformációk használata lehetővé teszi, hogy ne csak azokat a technológiákat használjuk modellünk feldolgozására melyek speciálisan az adott modellezési nyelvhez készültek hanem a modelleket megpróbáljuk átalakítani - lehetőleg a szemantikai tartalom megőrzésével és automatizáltan - egy olyan modellezési nyelvre amelyhez már létezik az általunk használni kívánt funkcionalitást támogató technológia.

Fontos megjegyezni, hogy előfordulhatnak olyan esetek is amikor a modell transzformáció komplexitása és elkészítésének költsége, ha egyáltalán lehetséges ilyet készíteni, nagyobb mintha speciálisan az adott nyelvhez készítenénk egy speciális eszközt ami megvalósítja az állvárt funkcionalitást.

\subsection{MagicDraw - Gamma transzformáció}

A MagicDraw beépülő modul fő célja SysML állapottérképek formális verifikációja, ehhez vagy megpróbálunk írni egy eszközt amely képes közvetlen ilyen állapottérképeken dolgozni vagy átalakítani egy olyan modellezési nyelvre amelyhez rendelkezik olyan eszköz ami ezt el tudja végezni. Ilyen lehetett volna például az \uppaal, ez viszont jóval alacsonyabb szintű mint a modellek amiket ellenőrizni szerettem volna ezért a transzformáció elvégzése potenciálisan nagyon komplex lett volna. Ilyen transzformációt azonban már készítettek egy állapottérkép modellező nyelv a Gamma és az \uppaal  között, ezért ahelyett, hogy közvetlen \uppaal modelleket készítenél Gamma modelleket transzformálok. Ezzel nem csak az \uppaal-ra való transzformálhatóságot és a formális verifikációt nyerem, hanem mindenm olyan funkciót ami a Gammához készül és a transzformációhoz szükséges megfeleltetések is egyszerűsödnek valamelyest.

\section{Verifikáció menete}

A verifikáció elvégzése a MagicDraw beépülő modul szemszögéből négy lépésből áll : 
\begin{itemize}
	\item MagicDraw modellek Gammává transzformálása
	\item Gamma modellek \uppaal modellé transzformálása (ezt a Gamma keretrendszer végzi el)
	\item tulajdonságok ellenőrzése UPPAAL segítségével
	\item Eredmény megjelenítése
\end{itemize}
Ezt a folyamatot \refstruc{fig:preliminaries-verif} szemlélteti. A Gamma által elvégzett lépéseket a zöld "Gammák" jelölik a modellben. Az ábrán a verifikációt a "Módosított Query Generátor" kezdeményezi, mely szintén a "Gamma" jelölést kapta. Ennek oka, hogy a beépülő modul ezen verziójában az ellenőrizendő tulajdonságok megadása még nem volt kiforrott ezért egy a Gamma keretrendszerből átemelt megoldás segítségével lehetett ezeket megadni és a verifikációt kezdeményezni. Jelen dolgozatban a továbbiakban szó lesz többek között ennek a kiváltásáról egy előnyösebb ebben a környezetben felhasználóbarátabb megoldással.

\begin{figure}[!ht]
	\centering
	\includegraphics[width=150mm, keepaspectratio]{figures/preliminaries/concept.png}
	\caption{Verifikáció menete a pluginban}
	\label{fig:preliminaries-verif}
\end{figure}
