%----------------------------------------------------------------------------
\chapter{\bevezetes}
%----------------------------------------------------------------------------

A IT technológiák térnyerésével egyre több és komplexebb rendszer készül, melyeknek sokszor valós időben kell működni. Mivel ilyen rendszerek jellemzően valamilyen biztonság kritikus környezetben működnek, elengedhetetlené válik ezek gondos megtervezése és átfogó vizsgálata különösen a helyes működés tekintetében.

A tervezés és ellenőrzés költséges, időigényes folyamat, ezért szükség van olyan eszközökre amelyek megkönnyítik vagy akár teljesen automatizálnak egyes folyamatokat. A tervezés során általában valamilyen modellvezért technikát alkalmaznak, melynek középpontjában a modellek állnak. A tervezés során elkészített tervek nagyon sok értékes információt tartalmaznak, melyeket újra fel tudunk használni és származtatni ezekből kódot, dokumentációt, vagy akár más modelleket, ezáltal időt és erőforrásokat megtakarítva. Ráadásul mivel ezeket automatikusan gépek végzik, minimalizálódnak az emberi hibák például a programkódban, ahhoz képest mintha ezeket kézzel végeznénk el.

Terveinket már érdemes a tervezés korai fázisaiban ellenőrizni, hiszen az itt vétett hibák akár kritikusak lehetnek a későbbiekben. Az ellenőrzésekhez szintén fel tudjuk használni a modelljeinket és szimulálni tudjuk a rendszert, vagy képesek vagyunk magát a modellt is vizsgálni formális módszerek segítségével.

A MagicDraw egy népszerű szoftver, és rendszer architektúra modellező eszköz ami fejlett grafikus interfészt nyújt a felhasználók számára. Modelleket elsősorban egy általános célú modellezési nyelvvel UML-el lehet készíteni, azonban UML profilok segítségével akár saját szakterület specifikus nyelvek használatára is lehetőségünk nyílik. Ilyen formában a MagicDraw lehetővé teszi modellek létrahozását SysML nyelven is amihez a profilt maga biztosítja.

Ugyan a MagicDraw számos fontos és hasznos funkcióval rendelkezik, még mindig megvan az igény újabbakra főleg Verifikáció/Validáció tekintetében. A MagicDrawToGamma nevű MagicDrawhoz készült plugin SysML állapottérképek formális verifikálásához nyújt megoldást, melyhez a Gamma Statechart Composition Frameworköt és az UPPAAL nevű eszközöket használja fel.

Az eszköz ugyan \emph{Proof of Concept} jelleggel már képes a verifikációt elvégezni, azonban, hogy az eszköz teljes értékű termékké válhasson, még fejlesztésre szorul különösen a felhasználói élmény tekintetében.

Az MSc Önálló laboratórium tantárgy keretében a MagicDrawToGamma továbbfejleszthetőségét vizsgáltam és végeztem a pluginon olyan változtatásokat amik jelentősen hozzá járulnak a plugin könnyebb felhasználásához, továbbá olyan plusz funkciók tervezésével, implementálással foglalkoztam, mint a kompozit modellezés támogatása vagy a verifikációhoz szükséges kifejezések megfogalmazásának és tárolásának lehetősége a modellben.
