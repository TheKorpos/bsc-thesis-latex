%----------------------------------------------------------------------------
\chapter{\bevezetes}
%----------------------------------------------------------------------------


%kontextus
Nagyméretű és komplex tervezése során szükségessé válik, hogy olyan fejlett modellező eszközök álljanak rendelkezésünkre, melyek nem csupán gördülékenyebbé teszik a modell alapú fejlesztés menetét, de funkcióikkal lehetővé teszik rendszerek átfogó vizsgálatát: validációját azaz szintaktikai ellenőrzését, verifikálását a működésének helyességének ellenőrzését. A modellekből lehet kódot generálni, ami meggyorsítja a fejlesztést és csökkenti az implementáció során vétett programozói hibák számát.

Modelleket érdemes rajzok, diagramok segítségével definiálni, ezek ugyanis könnyebben megérthetők és hibákat is könnyebb észrevenni, mint egy kód alapú leírás esetében. A legelterjedtebb diagram alapú modellezési nyelvek az UML és a SysML melyek absztrakciók alkalmazásával rendszerek implementációtól független leírását teszik lehetővé. Ezek vagy hasonló modellezési nyelvek alkalmazása egy ipari projekt esetében kiemelkedően fontosak, ugyanis lehetővé teszik a rendszerek elemzését hibák felderítése céljából. Ezek felfedezése és kiküszöbölése még a tervezési fázisban nem jár jelentős többletköltségekkel hiszen nem kell az implementációt megváltoztatni, termékeket visszahívni egy esetleg utólag felfedezett tervezési hiba miatt.

Az iparban egy széleskörű alkalmazott eszköz a MagicDraw, ami többek között UML és SysML nyelveken teszi lehetővé modellek készítését. Ezeken magas szinten leírható egy rendszer felépítése és működése. MagicDraw-t gyakran alkalmaznak kritikus rendszerek modellezésére. Ezek esetében elengedhetetlen, hogy matematikai precizitással megbizonyosodjunk működésének helyességéről, hiszen egy meghibásodás komoly anyagi veszteségekkel vagy akár ember életek veszélyeztetésével járhat. Bár a MagicDraw már sok hasznos funkcióval rendelkezik, formális verifikációt végrehajtására még nincs lehetőség. Az ipar szempontjából jelentős előnyökkel járna viselkedésmodellek formális verifikációjának támogatottsága.
%megoldás
A Gamma nevű eszköz a tanszéken fejlesztett állapottérképek modellezésére és ezek fejlesztésének támogatására készített eszköz, ami elsősorban Eclipses környezetekben alkalmazható. Ugyan a Gamma eszközkészlete nem annyira kiterjedt mint a MagicDraw-é, de az állapottérképek formális verifikációja a funkcióinak részét képezi. A Gamma saját nyelvet alkalmaz az állapottérképek leírására ami kifejezetten úgy lett megalkotva, hogy támogassa más nyelvek leképezhetőségét például Yakinduét. Dolgozatomban állapotgépek formális verifikációját azáltal teszem lehetővé, hogy egy leképzést biztosítok a MagicDraw és a Gamma modelljei között és a Gamma formális verifikációhoz kötődő funkcióit MagicDraw-n belül is végrehajthatóvá teszem.

Dolgozatomban bemutatom a Gamma és MagicDraw eszközöket és a leképzés megvalósításához szükséges háttérismereteket, részletesen kitérve azokra az alapvető ismeretekre és tehcnológiákra amelyek lehetővé teszik a feladat megvalósítását. Továbbá ismertetem azokat az elméleti is implementációs kihívásokat melyekkel a feladat végrehajtása során szembe kellett néznem és az ezekre alkalmazott megoldásaimat, ezek előnyeivel és hátrányait egybevetve.

%kapcsolódó mukák rész jön. am ilyan, hogy... ehhez hasonló munkák pl xy ami ezt csiája, z ami mást csinál Gamma
%többi eszköz formális verifikációra
%leges legutolsó
Hasonló munkák például maga a Gamma hasonló hiszen az eszköz már megvalósít egy leképzést saját nyelve: a már említett Yakindu - Gamma leképzést.

A dolgozat feléptése a következő: a \ref{chap:hatter}. fejezetben bemutatom a munkám megértéséhez szükséges alapismereteket, \aref{chap:contrib}. fejezetben ismertetem az alkalmazott megoldást elméleti és gyakorlati szempontból.


