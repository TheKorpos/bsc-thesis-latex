%----------------------------------------------------------------------------
\chapter{\bevezetes}
%----------------------------------------------------------------------------

A viselkedés modellezés célja a rendszer, vagy szoftver működésének leírása. A leírt rendszerek a modellezés során kiemelt aspektusok mentén nem csak dokumentálják az elvárt működést, de lehetőséget adnak ezek átfogó vizsgálatára, ellenőrzésére is. Az ellenőrzés többféle módszerrel történhet, például szimulációval és teszteléssel azonban ezek hatékonysága erősen függ a tesztkészlettől és a lefedettségtől. Egy másik megközelítés, formális matematikai módszerekkel való ellenőrzés, azaz formális verifikáció.

Viselkedés modellek leírására többféle lehetőségünk van egyik ilyen az állapottérkép. Az állapot térkép egy véges hierarchikus diszkrét állapotgép grafikus leírása. Az így definiált állapotgépek visszavezethetők alapvető matematikai formalizmusokra amelyeket hatékonyan lehet ellenőrizni hatékony algoritmusokkal így géppel.


