%----------------------------------------------------------------------------
\chapter{\bevezetes}
%----------------------------------------------------------------------------

A IT technológiák térnyerésével egyre több és komplexebb rendszer készül, melyeknek sokszor valós időben kell működni. Mivel ilyen rendszerek jellemzően valamilyen biztonság kritikus környezetben működnek, elengedhetetlené válik ezek gondos megtervezése és átfogó vizsgálata különösen a helyes működés tekintetében.

A tervezés és ellenőrzés költséges, időigényes folyamat, ezért szükség van olyan eszközökre amelyek megkönnyítik vagy akár teljesen automatizálnak egyes folyamatokat. A tervezés során általában valamilyen modellvezért technikát alkalmaznak, melynek középpontjában a modellek állnak. A tervezés során elkészített tervek nagyon sok értékes információt tartalmaznak, melyeket újra fel tudunk használni és származtatni ezekből kódot, dokumentációt, vagy akár más modelleket, ezáltal időt és erőforrásokat megtakarítva. Ráadásul mivel ezeket automatikusan gépek végzik, minimalizálódnak az emberi hibák például a programkódban, ahhoz képest mintha ezeket kézzel végeznénk el.

Terveinket már érdemes a tervezés korai fázisaiban ellenőrizni, hiszen az itt vétett hibák akár kritikusak lehetnek a későbbiekben. Az ellenőrzésekhez szintén fel tudjuk használni a modelljeinket és szimulálni tudjuk a rendszert, vagy képesek vagyunk magát a modellt is vizsgálni formális módszerek segítségével.

A MagicDraw egy mára de-facto ipari standarddá vált szoftver, és rendszer architektúra modellező eszköz ami fejlett grafikus interfészt nyújt a felhasználók számára. Modelleket elsősorban egy általános célú modellezési nyelvvel UML-el lehet készíteni, azonban UML profilok segítségével akár saját szakterület specifikus nyelvek használatára is lehetőségünk nyílik. Ilyen formában a MagicDraw lehetővé teszi modellek létrahozását SysML nyelven is amihez a profilt maga biztosítja. A dolgozat a továbbiakban SysML modellekkel foglalkozik.

Ugyan a MagicDraw számos fontos és hasznos funkcióval rendelkezik, még mindig megvan az igény újabbakra főleg Verifikáció/Validáció tekintetében. A MagicDrawToGamma nevű MagicDrawhoz készült plugin SysML állapottérképek formális verifikálásához nyújt megoldást, melyhez a Gamma Statechart Composition Frameworköt és az UPPAAL nevű eszközöket használja fel.

Az eszköz ugyan \emph{Proof of Concept} jelleggel már képes a verifikációt elvégezni, azonban, hogy akár szélesebb körben is használható eszközzé válhasson még sok tekintetben fejlesztésre szorul. Jelen dolgozat célja bemutatni azokat a fejlesztéseket amiket a mesterképzés során végeztem az eszközön és visszatekintve kiértékelni azokat a mérnöki megoldásokat melyeket a fejlesztés során hoztam.

A dolgozat felépítése a következő: a második fejezetben ismertetem azokat az ismereteket amelyek a dolgozat során felvetülő problémák illetve az ezekre adott megoldások megértéséhez szükségesek. A harmadik fejezetben ismertetem a projekt céljait és az ezekhez vezető utat, alkalmazott megoldásokat. A negyedik fejezetben egy példán bemutatom ez eszköz működését. Az ötödik fejezetben mérések segítségével megvizsgálom az alkalmazott technikai megoldások teljesítményét. Végül a hatodik fejezetben összegzem a munkám.