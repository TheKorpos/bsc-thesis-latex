\chapter{Kapcsolódó munkák}
\label{chap:related-work}
\section{Gamma Statechart Composition Framework}
A Gamma Statechart Composition Framework\cite{gammaf} egy tanszéken fejlesztett eszköz, amit komponenst alapú viselkedés modellezést tesz lehetővé. Az eszköz képes kódot generálni és a formális verifikáció végrehajtására, továbbá képes Yakindu modelleket áttranszformálni Gamma modellekké. A Yakindu-Gamma transzformáció megvalósítása hasonló mint az általam implementált MagicDraw-gamma transzformációé. A Gamma keretrendszerrel \aref{sec:gamma-framework} szakasz foglalkozik részletesebben. 

\section{TismTool}

TismTool\cite{veugen2012framework} egy keretrendszer többszálú, komponens alapú rendszerek fejlesztéséhez, ami képes UML modellekből kódot generálni különböző nyelveken (C\#, Java, C++ vagy C) és ezeket verifikálni. A TismTool képes különböző UML modellező eszközök modelljeit feldolgozni, az egyetlen megkötés, hogy ezek képesek legyenek az UML model XML Metadata Interchange (XMI) szabvány szerint modellt sorosítani a MagicDrawra ez teljesül.

Az eszköz azáltal végzi el a verifikációt, hogy kódot generál az XMI fájlokból amit egy futtatókörnyezettel kell végrehajtatni továbbá nem formális alapokon hajtja végre a verifikációt, hanem ún. futás idejű verifikációt használ\footnote{http://www.tismtool.com/tooling.html}.

%http://dre.sourceforge.net/




