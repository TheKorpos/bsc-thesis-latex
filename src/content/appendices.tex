%----------------------------------------------------------------------------
\appendix
%----------------------------------------------------------------------------
\chapter*{\fuggelek}\addcontentsline{toc}{chapter}{\fuggelek}
\setcounter{chapter}{\appendixnumber}
%\setcounter{equation}{0} % a fofejezet-szamlalo az angol ABC 6. betuje (F) lesz
\numberwithin{equation}{section}
\numberwithin{figure}{section}
\numberwithin{lstlisting}{section}
%\numberwithin{tabular}{section}

%----------------------------------------------------------------------------
\section{Megfeleltetések, áttekintés}
%----------------------------------------------------------------------------
%\begin{figure}[!ht]
%\centering
%\includegraphics[width=150mm, keepaspectratio]{figures/TeXstudio.png}
%\caption{A TeXstudio \LaTeX-szerkesztő.} 
%\end{figure}

\begin{table}[!ht]
	\centering
	\begin{tabular}{ l l l}
		MagicDraw & Gamma & Verifikálható \\ \hline
		Vertex (absztrakt) & StateNode (absztrakt) & - \\
		State & State & igen \\
		CompositeState & State, egy belső régió & igen \\
		OrthogonalState & State, több belső régió & igen \\
		InitialState & InitialState & igen \\
		TerminalState & nincs & - \\
		FinalState & állapot 'FinalState' néven & igen \\
		EntryPoint & nincs & - \\
		ExitPint & nincs & - \\
		Conn. Point Reference & nincs & - \\
		SubmachineState & nincs & - \\
		ForkState & ForkState & nem \\
		JoinState & JoinState & nem \\
		Choice State & Choice State & nem \\
		Junction & Merge & nem \\
		DeepHistory & DeepHistory & igen \\
		ShallowHistory & ShallowHistory & igen \\
		Region  & Region & igen \\
		Transition & Transition & igen \\
		Signal, SignalEvent & Event & igen \\
		Trigger, SignalEvent & EventTrigger & igen \\
		\multirow{2}{*}{Trigger, TimeEvent(rel)} & TimeoutDeclaration, SetTimeoutAction, & \multirow{2}{*}{igen} \\
		&TimeoutTrigger&\\
		Tigger, TimeEvent(abs) & nincs & - \\
		Property(Integer) & VariableDeclaration(IntegerSpec.) & igen \\
		Property(Boolean) & VariableDeclaration(BooleanSpec.) & igen \\
		Action, FunctionBehavior & SendSignalAction, AssignMentAction & igen \\
		Guard(OpaqueExpression) & Expression & igen
	\end{tabular}
\end{table}




