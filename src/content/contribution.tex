\chapter{Plugin továbbfejlesztése}

Ebben a felyezetben ismertetem azokat a fejlesztéseket amelyeket a mesterképzés során az eszközön végeztem.

\section{Fejlesztés céljai}

A fejlesztés három fő célt felé halad. Az egyik, hogy állapottérképeket ne csak izoláltan, egymagukban de egy komplex rendszer részeként lehessen ellenőrizni. Ebben a rendszerben komponensek kommunikálnak egymással a komponensek viselkedését pedig állapottérképek írják le.

%TODO felulbiralni
A második cél a felhasználói élmény javítás oly módon, hogy a rendszer tulajdonságainak leírásához biztosítson egy Computational Tree Logic nyelvtant és az ezen a nyelven leírt tulajdonságokat a modellben el lehessen tárolni.

A harmadik cél, hogy a \uppaal / Gamma által biztosított ellenpéldákból olyan szimuláció álljon elő ami képes szimulálni és a szimuláción keresztül megmutatni a felhasználónak azt az esemény sort amely sérti valamelyik tulajdonságát a modellnek.

\section{MagicDraw profil}

\section{Kompozíciók transzformációja}

Egy nagy komplex rendszert célszerű nem egyben, hanem részekre bontva modellezni majd a részek egymáshoz illesztéséből, komponálásából képezni a teljes rendszert. Állapottérképek dekomponálását azaz részekre bontását a Gamma is támogatja. A kihívás a SysML és a Gamma közötti megfeleltetések megválasztása oly módon, hogy a szemantika ne sérüljön. A megfeleltetés két szempontól kell vizsgálni, egyszer az elemek tartalmazási hierarchiái szerint, egyszer pedig a köztük modellezett kommunikáció szerint.

\subsection{Struktúra megfeleltetése}
A struktúrák megfeleltetésénél könnyebb dolgom volt, hiszen mind szintaktikailag mind szemantikailag a két nyelv nagyon hasonló. Ami viszont különbség, hogy a komponensek szinkron, illetve asszinkron volta Gammában explicit jelölt, %TODO 

\subsection{Kommunikáció megfeleltetése}

\section{CTL nyelvtan}

\section{Back-annotation}

\section{Szimuláció}

\section{Példa} %TODO TBD